\documentclass[a4paper,11pt]{article}

    %Use system fonts, only apply to XeTeX
    \usepackage{fontspec}
    \usepackage{xeCJK}
    \setmainfont[Ligatures={TeX}]{Minion Pro}
    \setsansfont[Ligatures={TeX}]{Myriad Pro}
    \setmonofont[Ligatures={TeX}]{Consolas}

    \setCJKmainfont[BoldFont=Adobe Heiti Std,ItalicFont=Adobe Kaiti Std,Ligatures={TeX}]{Adobe Song Std}
    \setCJKfamilyfont{kaiti}[Ligatures={TeX}]{Adobe Kaiti Std}
    \setCJKfamilyfont{sf}[Ligatures={TeX}]{Adobe Heiti Std}
    \setCJKfamilyfont{it}[Ligatures={TeX}]{Adobe Kaiti Std}
    \setCJKfamilyfont{tt}[Ligatures={TeX}]{Adobe Kaiti Std}

    \usepackage{unicode-math}
    \unimathsetup{math-style=TeX}
    \setmathfont{XITS Math}

    %Adjust margin of document
    \usepackage{geometry}
    \geometry{left=2.5cm,right=2.5cm,top=2.5cm,bottom=2.5cm}

    \setlength{\parskip}{12pt}
    \setlength{\parindent}{0pt}

    % Graphics with mpost
    \usepackage{graphicx}
    \DeclareGraphicsRule{*}{eps}{*}{}

    % Hyperref in pdf
    \usepackage[usenames,dvipsnames,table]{xcolor}
    \usepackage{hyperref}
    \hypersetup{breaklinks,colorlinks,linkcolor=RoyalBlue,citecolor=red,urlcolor=purple,pdfstartview=FitH,pdfauthor={坂本ポテコ},pdftitle={XiaoTianQuan Firmware Control Protocol}}

    \usepackage{makecell}
    \renewcommand\theadfont{\bfseries}

    \usepackage{xspace}
    
    \usepackage{tabularx}

    \newcommand{\iic}{I\textsuperscript{2}C\xspace}
    \newcommand{\tdes}{\thead{Description}}
    \newcommand{\tacc}{\hline\thead{Access}}

    \newcolumntype{Y}{>{\centering\arraybackslash}X}

    \usepackage{environ}    
    \NewEnviron{regdes}{
        \sffamily\begin{tabularx}{\textwidth}{|c|Y|Y|Y|Y|Y|Y|Y|Y|}
            \hline
            \thead{Bit}   &   7   &   6   &   5   &   4   &   3   &   2   &   1   &   0   \\
            \hline
            \BODY
            \hline
        \end{tabularx}
    }

    \newenvironment{field}[1]
    {
        \par\vspace{\baselineskip}
        \textbf{\textsf{\large #1}}
        \par
    }{}

    %\newcommand{\fields}{\vspace{2em}\textsf{\textbf{\Large Fields}}} 

\begin{document}

\title{\Huge{XiaoTianQuan Firmware\\ \vspace{1.5em} Control Protocol}}
\author{坂本ポテコ}
\maketitle
\clearpage

\tableofcontents
\clearpage


{\Huge{Work In Progress.}}

\section{Supported Transport Protocols}
Currently only \iic protocol is supported. Serial is planned.

\section{\iic Protocol}
\subsection{Registers}

\subsubsection{Product Release Control RC0-RC9}

Address: 0x10

\begin{regdes}
\tdes & S8 & S7 & S6 & S5 & S4 & S3 & S2 & S1 \\
\tacc & W  & W  & W  & W  & W  & W  & W  & W  \\
\end{regdes}

\begin{field}{S1-8}
    Write 1 to start releasing product in slot. If there's multiple bits set, the least significant 1 bit will be used.
\end{field}

\subsubsection{Product Release Status PRS0-RS9}

\begin{regdes}
\tdes & S8 & S7 & S6 & S5 & S4 & S3 & S2 & S1 \\
\tacc & R/W& R/W& R/W& R/W& R/W& R/W& R/W& R/W \\
\end{regdes}

\begin{field}{S1-S8}
    When read, 1 indicates last release was successful, 0 indicates there's no release or release failed. Write 1 to clear the bit.
\end{field}

\subsubsection{Product Release Error RE}

\begin{regdes}
\tdes & S8 & S7 & S6 & S5 & S4 & S3 & S2 & S1 \\
\tacc & R/W& R/W& R/W& R/W& R/W& R/W& R/W& R/W \\
\end{regdes}


\subsubsection{Power Control}



\end{document}


















