\documentclass[a4paper,11pt]{article}

    %Use system fonts, only apply to XeTeX
    \usepackage{fontspec}
    \usepackage{xeCJK}
    \setmainfont[Ligatures={TeX}]{Minion Pro}
    \setsansfont[Ligatures={TeX}]{Myriad Pro}
    \setmonofont[Ligatures={TeX}]{Consolas}

    \setCJKmainfont[BoldFont=Adobe Heiti Std,ItalicFont=Adobe Kaiti Std,Ligatures={TeX}]{Adobe Song Std}
    \setCJKfamilyfont{kaiti}[Ligatures={TeX}]{Adobe Kaiti Std}
    \setCJKfamilyfont{sf}[Ligatures={TeX}]{Adobe Heiti Std}
    \setCJKfamilyfont{it}[Ligatures={TeX}]{Adobe Kaiti Std}
    \setCJKfamilyfont{tt}[Ligatures={TeX}]{Adobe Kaiti Std}

    \usepackage{unicode-math}
    \unimathsetup{math-style=TeX}
    \setmathfont{XITS Math}

    %Adjust margin of document
    \usepackage{geometry}
    \geometry{left=2.5cm,right=2.5cm,top=2.5cm,bottom=2.5cm}

    \setlength{\parskip}{12pt}
    \setlength{\parindent}{0pt}

    % Graphics with mpost
    \usepackage{graphicx}
    \DeclareGraphicsRule{*}{eps}{*}{}

    % Hyperref in pdf
    \usepackage[usenames,dvipsnames,table]{xcolor}
    \usepackage{hyperref}
    \hypersetup{breaklinks,colorlinks,linkcolor=RoyalBlue,citecolor=red,urlcolor=purple,pdfstartview=FitH,pdfauthor={坂本ポテコ},pdftitle={XiaoTianQuan Firmware Control Protocol}}

    \usepackage{makecell}
    \renewcommand\theadfont{\bfseries}

    \usepackage{xspace}

    \usepackage{tabularx}

    \newcommand{\iic}{I\textsuperscript{2}C\xspace}
    \newcommand{\addr}[1]{\textsf{\textbf{Address} #1}}
    \newcommand{\offset}[1]{\textsf{\textbf{Offset} #1}}
    
    \newcommand{\tdes}{\thead{Description}}
    \newcommand{\tacc}{\hline\thead{Access}}
    
    

    \newcolumntype{Y}{>{\centering\arraybackslash}X}

    \usepackage{environ}
    \NewEnviron{regdes}{
        \sffamily\begin{tabularx}{\textwidth}{|c|Y|Y|Y|Y|Y|Y|Y|Y|}
            \hline
            \thead{Bit}   &   7   &   6   &   5   &   4   &   3   &   2   &   1   &   0   \\
            \hline
            \BODY
            \hline
        \end{tabularx}
    }

    \newenvironment{field}[1]
    {
        \par\vspace{\baselineskip}
        \textbf{\textsf{\large #1}}
        \par
    }{}

    %\newcommand{\fields}{\vspace{2em}\textsf{\textbf{\Large Fields}}}

\begin{document}

\title{\Huge{XiaoTianQuan Firmware\\ \vspace{1.5em} Control Protocol}}
\author{坂本ポテコ}
\maketitle
\clearpage

\tableofcontents
\clearpage


{\Huge{Work In Progress.}}

\section{Supported Transport Protocols}
    Currently only \iic protocol is supported. Serial is planned.

\section{\iic Protocol}
\subsection{Registers}

\subsubsection{Product Release Control, RC0-RC15}

    This register controls the slot to release the product.

    \addr{0x10}

    \offset{0-F}

    \begin{regdes}
    \tdes & S8 & S7 & S6 & S5 & S4 & S3 & S2 & S1 \\
    \tacc & W  & W  & W  & W  & W  & W  & W  & W  \\
    \end{regdes}

    \begin{field}{S1-8}
        Write 1 to start releasing product in slot. If there's multiple bits set, the least significant bit will be used.
    \end{field}

\subsubsection{Product Release Status, PRS0-RS15}

    This register is the status of the slot of last release.

    \addr{0x20}

    \offset{0-F}

    \begin{regdes}
    \tdes & S8 & S7 & S6 & S5 & S4 & S3 & S2 & S1 \\
    \tacc & R  & R  & R  & R  & R  & R  & R  & R  \\

    \end{regdes}

    \begin{field}{S1-S8}
        0 indicates last release was successful or no release, 1 indicates the release failed.
    \end{field}

\subsubsection{Product Release Status Slot, RSS}\label{reg:rss}

    \addr{0x30}

    This register holds the slot ID of \ref{reg:re} \verb|RE|. When written, contents of \verb|RE| is changed to the slot ID of \verb|RSS|.

    \begin{regdes}
    \tdes & \multicolumn{8}{c|}{Slot ID} \\
    \tacc & \multicolumn{8}{c|}{R/W} \\
    \end{regdes}

    \begin{field}{Slot ID}
        The slot ID for register RSS.
    \end{field}

\subsubsection{Product Release Error, RE}\label{reg:re}

    \addr{0x31}

    This register holds the error information of the slot in \ref{reg:rss} \verb|RSS|..

    \begin{regdes}
    \tdes & \multicolumn{8}{c|}{Error ID} \\
    \tacc & \multicolumn{8}{c|}{R} \\
    \end{regdes}

    \begin{field}{Error ID}
        The error ID of the corresponding register.
    \end{field}


\subsubsection{Power Control, PWR}

    \addr{0x80}

    \begin{regdes}
    \tdes & \multicolumn{6}{c|}{Reserved} & Slp& AppPwr \\
    \tacc & \multicolumn{6}{c|}{R} & R/W& R/W \\
    \end{regdes}

    \begin{field}{Slp}
        Write 1 to enter sleep mode.
    \end{field}

    \begin{field}{AppPwr}
        Write 1 to turn off power of app board.
    \end{field}

\subsubsection{Battery Voltage, BAT}

    These register holds the FP32 value battery voltage in Volt.
    
    \addr{0x81}
    
    \begin{regdes}
    \tdes & \multicolumn{8}{c|}{Low 16 bits of battery voltage in FP32} \\
    \tacc & \multicolumn{8}{c|}{R} \\
    \end{regdes}

    \addr{0x82}

    \begin{regdes}
    \tdes & \multicolumn{8}{c|}{High 16 bits of battery voltage in FP32} \\
    %\tacc & R/W& R/W& R/W& R/W& R/W& R/W& R/W& R/W \\
    \tacc & \multicolumn{8}{c|}{R} \\
    \end{regdes}


\end{document}


















